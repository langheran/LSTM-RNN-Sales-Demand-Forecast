\documentclass[]{article}
\usepackage{lmodern}
\usepackage{amssymb,amsmath}
\usepackage{ifxetex,ifluatex}
\usepackage{fixltx2e} % provides \textsubscript
\ifnum 0\ifxetex 1\fi\ifluatex 1\fi=0 % if pdftex
  \usepackage[T1]{fontenc}
  \usepackage[utf8]{inputenc}
\else % if luatex or xelatex
  \ifxetex
    \usepackage{mathspec}
  \else
    \usepackage{fontspec}
  \fi
  \defaultfontfeatures{Ligatures=TeX,Scale=MatchLowercase}
\fi
% use upquote if available, for straight quotes in verbatim environments
\IfFileExists{upquote.sty}{\usepackage{upquote}}{}
% use microtype if available
\IfFileExists{microtype.sty}{%
\usepackage{microtype}
\UseMicrotypeSet[protrusion]{basicmath} % disable protrusion for tt fonts
}{}
\usepackage[margin=1in]{geometry}
\usepackage{hyperref}
\PassOptionsToPackage{usenames,dvipsnames}{color} % color is loaded by hyperref
\hypersetup{unicode=true,
            pdftitle={Final Project Proposal},
            pdfauthor={Nisim Hurst},
            colorlinks=true,
            linkcolor=Maroon,
            citecolor=Blue,
            urlcolor=blue,
            breaklinks=true}
\urlstyle{same}  % don't use monospace font for urls
\usepackage{longtable,booktabs}
\usepackage{graphicx,grffile}
\makeatletter
\def\maxwidth{\ifdim\Gin@nat@width>\linewidth\linewidth\else\Gin@nat@width\fi}
\def\maxheight{\ifdim\Gin@nat@height>\textheight\textheight\else\Gin@nat@height\fi}
\makeatother
% Scale images if necessary, so that they will not overflow the page
% margins by default, and it is still possible to overwrite the defaults
% using explicit options in \includegraphics[width, height, ...]{}
\setkeys{Gin}{width=\maxwidth,height=\maxheight,keepaspectratio}
\IfFileExists{parskip.sty}{%
\usepackage{parskip}
}{% else
\setlength{\parindent}{0pt}
\setlength{\parskip}{6pt plus 2pt minus 1pt}
}
\setlength{\emergencystretch}{3em}  % prevent overfull lines
\providecommand{\tightlist}{%
  \setlength{\itemsep}{0pt}\setlength{\parskip}{0pt}}
\setcounter{secnumdepth}{0}
% Redefines (sub)paragraphs to behave more like sections
\ifx\paragraph\undefined\else
\let\oldparagraph\paragraph
\renewcommand{\paragraph}[1]{\oldparagraph{#1}\mbox{}}
\fi
\ifx\subparagraph\undefined\else
\let\oldsubparagraph\subparagraph
\renewcommand{\subparagraph}[1]{\oldsubparagraph{#1}\mbox{}}
\fi

%%% Use protect on footnotes to avoid problems with footnotes in titles
\let\rmarkdownfootnote\footnote%
\def\footnote{\protect\rmarkdownfootnote}

%%% Change title format to be more compact
\usepackage{titling}

% Create subtitle command for use in maketitle
\newcommand{\subtitle}[1]{
  \posttitle{
    \begin{center}\large#1\end{center}
    }
}

\setlength{\droptitle}{-2em}
  \title{Final Project Proposal}
  \pretitle{\vspace{\droptitle}\centering\huge}
  \posttitle{\par}
\subtitle{A price elasticity-based LSTM prediction model}
  \author{\href{mailto:langheran@gmail.com}{Nisim Hurst}}
  \preauthor{\centering\large\emph}
  \postauthor{\par}
  \predate{\centering\large\emph}
  \postdate{\par}
  \date{Wednesday 25 April 2018}

\usepackage{float} \usepackage{caption}
\makeatletter\renewcommand*{\fps@figure}{H}\makeatother

\usepackage{amsthm}
\newtheorem{theorem}{Theorem}
\newtheorem{lemma}{Lemma}
\theoremstyle{definition}
\newtheorem{definition}{Definition}
\newtheorem{corollary}{Corollary}
\newtheorem{proposition}{Proposition}
\theoremstyle{definition}
\newtheorem{example}{Example}
\theoremstyle{definition}
\newtheorem{exercise}{Exercise}
\theoremstyle{remark}
\newtheorem*{remark}{Remark}
\newtheorem*{solution}{Solution}
\begin{document}
\maketitle

\textbf{Profesor:} \href{mailto:vallejo@itesm.mx}{Edgar Emmanuel Vallejo
Clemente}

\textbf{Team Members:} \href{mailto:A01012491@itesm.mx}{Nisim Hurst,
A01012491}

\subsection{Problem description}\label{problem-description}

Densely connected simple neural networks are good when the i.d.d.
(independent and identically distributed) assumption holds on the sample
data. However, data that is correlated in time or space is a bit
trickier to treat.

One example of these kind of task is when you try to predict product
sell quantities from price. Right price strategies are difficult to
predict because most of the product are subject to behavioral economics
variables like the
\href{https://en.wikipedia.org/wiki/Veblen_good}{Veblen Effect} or
changes in demographics that are outside of the scope of the data.

Price elasticity models are usually designed by a human and optimized by
a machine, but nowadays these models are falling short behind the
velocity of global economic changes. Thus, we propose a machine
learning-based method that could cope with those changes.

Namely, we would like to estimate units sold using a learned price
elasticity model.

\subsection{Solution proposal}\label{solution-proposal}

Hidden Markov models fail to capture long term relationships between the
data because of the Markov assumption that a future state is only
dependent of the immediately previous state. Also, on Markov chains you
have to find a way to compress the whole problem into one variable only.

Thus, we propose will try out a long short-term memory recurrent neural
network to be able to capture relationships that go over 1 year long.

\subsection{Dataset description}\label{dataset-description}

The proposed dataset was proposed by Dr.~Laura Hervert on and is
currently being used on the Research Methods class:

\begin{enumerate}
\def\labelenumi{\arabic{enumi}.}
\tightlist
\item
  \textbf{Dimensions.} 78578 samples, 15 features, 1030 unique sku
  (sequences), 70 categories. About 14 sku ssales sequences per
  category. The data is spread out evenly over a period of roughly 2.5
  years, i.e.~missing data over the sequences are trifling.
\item
  \textbf{Variables.} sku, category (subjective), time (week precision),
  prices, cost, units sold, temperature ranges, rain and whether or not
  the sku had a promotion.
\item
  \textbf{Description.} Here is a description taken from the original
  dataset page:

  \begin{enumerate}
  \def\labelenumii{\arabic{enumii}.}
  \tightlist
  \item
    Category- Classification for things regarded as having particular
    shared characteristics
  \item
    SKU -- Products
  \item
    Year -- Year in which the sale occurred
  \item
    Week -- Week of the year in which the sale occurred
  \item
    System price -- Price in the information system of the company at
    the time of sale
  \item
    Competitor's price - Price of the principal competitor's at the time
    of sale
  \item
    Net cost - Cost of the product without considering the profit margin
  \item
    Units -- Sold units during the week
  \item
    Sales amount - Income obtained from the sales during the week
  \item
    Real price - Income obtained from the sale divided by the sold units
  \item
    Min temp - Minimum temperature recorded in the week of sale
  \item
    Avg temp - Average temperature recorded in the week of sale
  \item
    Max temp - Maximun temperature recorded in the week of sale
  \item
    Rain intensity -- Scale to define the intensity of the rain during
    the week of sale
  \item
    Promotion -- Binary value 1- for a sale with a promotion involved 0
    -- for a regular sale.
  \end{enumerate}
\end{enumerate}

\subsection{Questions to answer}\label{questions-to-answer}

\begin{enumerate}
\def\labelenumi{\arabic{enumi}.}
\tightlist
\item
  \textbf{Can this dataset be used to train a recurrent neural network
  that produces over 60\% accuracy on the predicted 1 month horizon?} Is
  the data sufficient? are there any bayes error bounds implicit in the
  data?
\item
  \textbf{How similar products sales relate to the data?} can we use
  data of one product category (human labeled) and use it to predict the
  units sold of another category? can we use a kind of transfer learning
  from other pre-trained datasets?
\item
  \textbf{Is there any seasonality for some products?} are there any
  periodic elements that would affect the units sold regardless of the
  other variables.
\item
  \textbf{How elastic are the products?} how changes in prices and other
  ancillary variables affect the units sold.
\end{enumerate}

\subsection{Additional comments}\label{additional-comments}

\begin{enumerate}
\def\labelenumi{\arabic{enumi}.}
\tightlist
\item
  Some holyday weekends have no data, we could use a connect-the-dots
  technique or weighted moving average on those days for completing the
  information cube.
\end{enumerate}


\end{document}
